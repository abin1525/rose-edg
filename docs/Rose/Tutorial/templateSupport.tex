\chapter{Template Support}

    This chapter is specific to demonstrating the C++ template support in ROSE. 
{\em This section is not an introduction to the general subject of C++ templates.} 
ROSE provides special handling for C++ templates because template instantiation 
must be controlled by the compiler.

   Templates that require instantiation are instantiated by ROSE and 
can be seen in the traversal of the AST (and transformed).  Any templates
that can be instantiated by the backend compiler {\bf and} {\em are not transformed}
are not output within the code generation phase.

\fixme{Provide a list of when templates are generated internally in the AST 
and when template instantiations are output.}

\section{Example Template Code \#1}
     
   This section presents figure~\ref{Tutorial:exampleTemplate1}, a simple 
C++ source code using a template. It is used as a basis for showing how 
template instantiations are handled within ROSE.

\begin{figure}[!h]
{\indent
{\mySmallFontSize


% Do this when processing latex to generate non-html (not using latex2html)
\begin{latexonly}
   \lstinputlisting{\TutorialExampleDirectory/inputCode_templateExample1.C}
\end{latexonly}

% Do this when processing latex to build html (using latex2html)
\begin{htmlonly}
   \verbatiminput{\TutorialExampleDirectory/inputCode_templateExample1.C}
\end{htmlonly}

% end of scope in font size
}
% End of scope in indentation
}
\caption{Example source code showing use of a C++ template.}
\label{Tutorial:exampleTemplate1}
\end{figure}

\begin{figure}[!h]
{\indent
{\mySmallFontSize


% Do this when processing latex to generate non-html (not using latex2html)
\begin{latexonly}
   \lstinputlisting{\TutorialExampleBuildDirectory/rose_inputCode_templateExample1.C}
\end{latexonly}

% Do this when processing latex to build html (using latex2html)
\begin{htmlonly}
   \verbatiminput{\TutorialExampleBuildDirectory/rose_inputCode_templateExample1.C}
\end{htmlonly}

% end of scope in font size
}
% End of scope in indentation
}
\caption{Example source code after processing using identityTranslator 
(shown in figure~\ref{Tutorial:exampleIdentityTranslator}).}
\label{Tutorial:exampleTemplate1}
\end{figure}



\section{Example Template Code \#2}
     
   This section presents figure~\ref{Tutorial:exampleTemplate1}, a simple 
C++ source code using a template function. It is used as a basis for showing how 
template instantiations are handled within ROSE.

\begin{figure}[!h]
{\indent
{\mySmallFontSize


% Do this when processing latex to generate non-html (not using latex2html)
\begin{latexonly}
   \lstinputlisting{\TutorialExampleDirectory/inputCode_templateExample2.C}
\end{latexonly}

% Do this when processing latex to build html (using latex2html)
\begin{htmlonly}
   \verbatiminput{\TutorialExampleDirectory/inputCode_templateExample2.C}
\end{htmlonly}

% end of scope in font size
}
% End of scope in indentation
}
\caption{Example source code showing use of a C++ template.}
\label{Tutorial:exampleTemplate2}
\end{figure}

\begin{figure}[!h]
{\indent
{\mySmallFontSize


% Do this when processing latex to generate non-html (not using latex2html)
\begin{latexonly}
   \lstinputlisting{\TutorialExampleBuildDirectory/rose_inputCode_templateExample2.C}
\end{latexonly}

% Do this when processing latex to build html (using latex2html)
\begin{htmlonly}
   \verbatiminput{\TutorialExampleBuildDirectory/rose_inputCode_templateExample2.C}
\end{htmlonly}

% end of scope in font size
}
% End of scope in indentation
}
\caption{Example source code after processing using identityTranslator 
(shown in figure~\ref{Tutorial:exampleIdentityTranslator}).}
\label{Tutorial:exampleTemplate1}
\end{figure}



