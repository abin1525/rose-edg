\chapter{Database Support}

    This chapter is specific to support in ROSE for persistent storage.
ROSE uses the SQLite database and makes it simple to store data in the
database for retrieval in later phases of processing large multiple file projects.

\fixme{Need more information here.}

\section{ROSE DB Support for Persistent Analysis}

   This section presents figure~\ref{Tutorial:exampleDataBase1}, a simple 
C++ source code using a template. It is used as a basis for showing how 
template instantiations are handled within ROSE. An example translator
using a database connection to store function information is shown in 
Fig.\ref{Tutorial:exampleDataBaseTranslator1} 
and Fig.\ref{Tutorial:exampleDataBaseTranslator2}. The output by the
translator operating on the C++ source code is shown in Fig. \ref{Tutorial:exampleTemplate1Output}. 


%---------------------------------------------------------
\begin{figure}[!h]
{\indent
{\mySmallFontSize


% Do this when processing latex to generate non-html (not using latex2html)
\begin{latexonly}
   \lstinputlisting{\TutorialExampleBuildDirectory/dataBaseUsage.aa}
\end{latexonly}

% Do this when processing latex to build html (using latex2html)
\begin{htmlonly}
   \verbatiminput{\TutorialExampleBuildDirectory/dataBaseUsage.aa}
\end{htmlonly}

% end of scope in font size
}
% End of scope in indentation
}
\caption{Example translator (part 1) using database connection to store function names.}
\label{Tutorial:exampleDataBaseTranslator1}
\end{figure}

%---------------------------------------------------------
\begin{figure}[!h]
{\indent
{\mySmallFontSize


% Do this when processing latex to generate non-html (not using latex2html)
\begin{latexonly}
   \lstinputlisting{\TutorialExampleBuildDirectory/dataBaseUsage.ab}
\end{latexonly}

% Do this when processing latex to build html (using latex2html)
\begin{htmlonly}
   \verbatiminput{\TutorialExampleBuildDirectory/dataBaseUsage.ab}
\end{htmlonly}

% end of scope in font size
}
% End of scope in indentation
}
\caption{Example translator (part 2) using database connection to store function names.}
\label{Tutorial:exampleDataBaseTranslator2}
\end{figure}

%---------------------------------------------------------
\begin{figure}[!h]
{\indent
{\mySmallFontSize


% Do this when processing latex to generate non-html (not using latex2html)
\begin{latexonly}
   \lstinputlisting{\TutorialExampleDirectory/inputCode_dataBaseExample1.C}
\end{latexonly}

% Do this when processing latex to build html (using latex2html)
\begin{htmlonly}
   \verbatiminput{\TutorialExampleDirectory/inputCode_dataBaseExample1.C}
\end{htmlonly}

% end of scope in font size
}
% End of scope in indentation
}
\caption{Example source code used as input to database example.}
\label{Tutorial:exampleDataBase1}
\end{figure}

%---------------------------------------------------------
\begin{figure}[!h]
{\indent
{\mySmallFontSize


% Do this when processing latex to generate non-html (not using latex2html)
\begin{latexonly}
   \lstinputlisting{\TutorialExampleBuildDirectory/dataBaseExample1.out}
\end{latexonly}

% Do this when processing latex to build html (using latex2html)
\begin{htmlonly}
   \verbatiminput{\TutorialExampleBuildDirectory/dataBaseExample1.out}
\end{htmlonly}

% end of scope in font size
}
% End of scope in indentation
}
\caption{Output from processing input code through database example
dataBaseTranslator\ref{Tutorial:exampleDataBaseTranslator1}.}
\label{Tutorial:exampleTemplate1Output}
\end{figure}


\section{Call Graph for Multi-file Application}

   This section shows an example of the use of the ROSE Database mechanism
where information is stored after processing each file as part of generating the
call graph for a project consisting of multiple files.  The separate files
are show in figures~\ref{Tutorial:exampleDataBase1} and ~\ref{Tutorial:exampleDataBase2}.
These files are processed using the translator in figure~\ref{Tutorial:exampleDataBase2}
to generate the final project call graph shown in figure~\ref{Tutorial:exampleDataBase2}.

\fixme{This example still needs to be implemented to use the new ROSE call graph generator.}


\section{Class Hierarchy Graph}

   This section presents a translator in figure~\ref{Tutorial:exampleDataBase2}, to
generate the class hierarchy graph of the example shown in
figure~\ref{Tutorial:exampleDataBase2}. The input is a multi-file application
show in figure~\ref{Tutorial:exampleDataBase2} and 
figure~\ref{Tutorial:exampleDataBase2}. {\em This example is incomplete.}

\fixme{This example is still incomplete.}

\commentout{
%---------------------------------------------------------
\begin{figure}[!h]
{\indent
{\mySmallFontSize


% Do this when processing latex to generate non-html (not using latex2html)
\begin{latexonly}
   \lstinputlisting{\TutorialExampleDirectory/inputCode_templateExample2.C}
\end{latexonly}

% Do this when processing latex to build html (using latex2html)
\begin{htmlonly}
   \verbatiminput{\TutorialExampleDirectory/inputCode_templateExample2.C}
\end{htmlonly}

% end of scope in font size
}
% End of scope in indentation
}
\caption{Example source code showing use of a C++ template.}
\label{Tutorial:exampleDataBase2}
\end{figure}

\begin{figure}[!h]
{\indent
{\mySmallFontSize


% Do this when processing latex to generate non-html (not using latex2html)
\begin{latexonly}
   \lstinputlisting{\TutorialExampleBuildDirectory/rose_inputCode_templateExample2.C}
\end{latexonly}

% Do this when processing latex to build html (using latex2html)
\begin{htmlonly}
   \verbatiminput{\TutorialExampleBuildDirectory/rose_inputCode_templateExample2.C}
\end{htmlonly}

% end of scope in font size
}
% End of scope in indentation
}
\caption{Example source code after processing using 
identityTranslator\ref{Tutorial:exampleIdentityTranslator}.}
\label{Tutorial:exampleTemplate1}
\end{figure}
}


