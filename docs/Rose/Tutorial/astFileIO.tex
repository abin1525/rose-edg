\chapter{AST File I/O}

   Figure~\ref{Tutorial:example_astFileIO} shows an
example of how to use the AST File I/O mechanism.  This chapter 
presents an example translator to write out an AST to a file and 
then read it back in.


\section{Source Code for File I/O}

    Figure~\ref{Tutorial:example_astFileIO}
shows an example translator which reads an input application, forms the
AST, writes out the AST to a file, then deletes the AST and reads the
AST from the previously written file. 

The input code is shown in figure~\ref{Tutorial:exampleInputCode_astFileIO},
the output of this code is shown in 
figure~\ref{Tutorial:exampleOutput_astFileIO}.

\begin{figure}[!h]
{\indent
{\mySmallFontSize

% Do this when processing latex to generate non-html (not using latex2html)
\begin{latexonly}
   \lstinputlisting{\TutorialExampleDirectory/inlineTransformations.C}
\end{latexonly}

% Do this when processing latex to build html (using latex2html)
\begin{htmlonly}
   \verbatiminput{\TutorialExampleDirectory/inlineTransformations.C}
\end{htmlonly}

% end of scope in font size
}
% End of scope in indentation
}
\caption{Example source code showing how to use the AST file I/O support.}
\label{Tutorial:example_astFileIO}
\end{figure}



\section{Input to Demonstrate File I/O}

   Figure~\ref{Tutorial:exampleInputCode_astFileIO}
shows the example input used for demonstration of the AST file I/O.
In this case we are reusing the example used in the inlining example.

\begin{figure}[!h]
{\indent
{\mySmallFontSize

% Do this when processing latex to generate non-html (not using latex2html)
\begin{latexonly}
   \lstinputlisting{\TutorialExampleDirectory/inputCode_inlineTransformations.C}
\end{latexonly}

% Do this when processing latex to build html (using latex2html)
\begin{htmlonly}
   \verbatiminput{\TutorialExampleDirectory/inputCode_inlineTransformations.C}
\end{htmlonly}

% end of scope in font size
}
% End of scope in indentation
}
\caption{Example source code used as input to demonstrate the AST file I/O support.}
\label{Tutorial:exampleInputCode_astFileIO}
\end{figure}



\section{Output from File I/O}

   Figure~\ref{Tutorial:exampleOutput_astFileIO} 
shows the output from the example file I/O tutorial example.

\begin{figure}[!h]
{\indent
{\mySmallFontSize

% Do this when processing latex to generate non-html (not using latex2html)
\begin{latexonly}
   \lstinputlisting{\TutorialExampleBuildDirectory/astFileIO_GenerateBinaryFile.out}
\end{latexonly}

% Do this when processing latex to build html (using latex2html)
\begin{htmlonly}
   \verbatiminput{\TutorialExampleBuildDirectory/astFileIO_GenerateBinaryFile.out}
\end{htmlonly}

% end of scope in font size
}
% End of scope in indentation
}
\caption{Output of input code after inlining transformations.}
\label{Tutorial:exampleOutput_astFileIO}
\end{figure}

\section{Final Code After Passing Through File I/O}

   Figure~\ref{Tutorial:exampleOutput_astFileIOSource}
shows the same file as the input demonstrating that the file I/O
didn't change the resulting generated code.  
{\em Much more sophisticated tests are applied internally to verify 
the correctness of the AST after AST file I/O.}

\begin{figure}[!h]
{\indent
{\mySmallFontSize

% Do this when processing latex to generate non-html (not using latex2html)
\begin{latexonly}
   \lstinputlisting{\TutorialExampleBuildDirectory/rose_inputCode_inlineTransformations.C}
\end{latexonly}

% Do this when processing latex to build html (using latex2html)
\begin{htmlonly}
   \verbatiminput{\TutorialExampleBuildDirectory/rose_inputCode_inlineTransformations.C}
\end{htmlonly}

% end of scope in font size
}
% End of scope in indentation
}
\caption{Output of input code after file I/O.}
\label{Tutorial:exampleOutput_astFileIOSource}
\end{figure}







